\paragraph{Theorem} Sturm–Liouville has real eigenvalues and  orthogonal eigenfunctions. In case of divided bound conditions $X$, each eigenvalue is simple. In any case, multiplicity of each eigenvalue is not greater than 2.
\subparagraph{Proof}
For each there are two independent solutions $u_1^\lambda$, $u_2^\lambda$. If $\lambda$ is eigenvalue, exist $A,B \in \mathbb{R}$ such that
$$\begin{cases}
B_a(Au_1^\lambda + Bu_2^\lambda) = 0\\
B_b(Au_1^\lambda + Bu_2^\lambda) = 0\\
\end{cases}$$
$$\begin{cases}
A\qty(\alpha u_1^\lambda(a)+\beta{u^\lambda_1}'(a)) + B\qty(\alpha u_2^\lambda(a)+\beta{u^\lambda_2}'(a)) = 0\\
A\qty(\gamma u_1^\lambda(b)+\delta {u^\lambda_1}'(b)) + B\qty(\gamma u_2^\lambda(b)+\delta {u^\lambda_2}'(b)) = 0\\
\end{cases}$$

If this linear system has solution, 
$$\qty(\alpha u_1^\lambda(a)+\beta{u^\lambda_1}'(a))\qty(\gamma u_2^\lambda(b)+\delta {u^\lambda_2}'(b)) - \qty(\alpha u_2^\lambda(a)+\beta{u^\lambda_2}'(a))\qty(\gamma u_1^\lambda(b)+\delta {u^\lambda_1}'(b)) = 0$$
This is actually equation in $\lambda$, that has infinite number of solution.


For $\lambda_1 \neq \lambda_2$,
$$Lu_1 = \lambda_1 u_1$$
$$Lu_2 = \lambda_2 u_2$$
$$\langle u_2, Lu_1 \rangle = \lambda_1\langle u_1, u_2 \rangle $$
$$\langle u_2, Lu_1 \rangle  = \langle u_1, Lu_2 \rangle = \lambda_2\langle u_1, u_2 \rangle $$
$$(\lambda_1-\lambda_2)\langle u_1, u_2 \rangle = 0 $$
Thus if $\lambda_1 \neq \lambda_2$, $\langle u_1, u_2 \rangle = 0 $.

Suppose $\lambda \in \mathbb{C}$:
$$Lu = \lambda u$$
$$\overline{Lu} = \overline{\lambda u}$$
$$\bar{L} \bar{u} = L \bar{u} = \bar{\lambda}\bar{ u}$$
Thus both $u$ and $\bar{u}$ are eigenfunctions, with different eigenvalues, i.e., $\langle u, \bar{u} \rangle = 0 $, meaning 
$$\int_a^b \abs{u}^2 \dd{x} = 0$$
$$u=0$$

$$Lu = Lu_1 + iLu_2$$
$$\overline{Lu }= Lu_1 - iLu_2 = L(u_1-iu_2) = L\bar{u}$$

Now
$$Lu = \lambda u$$
$$Lv = \lambda v$$
$$0 = vLu - uLv = \frac{1}{2} \qty[\qty[(pu')'+qu]v - \qty[(pv')'+qv]u] = \frac{1}{2} \qty[p\cdot(u'v-v'u)]' = 0$$
Thus
$$p\cdot(u'v-v'u) = c = \text{const}$$
$$u'v-v'u = \frac{c}{p}$$
Since $p>0$ and $u'v-v'u$ is $0$ on bound, $c=0$, thus 
$$u'v = v'u $$
i.e.,
$\qty(\frac{u}{v})'$ or 
$\qty(\frac{v}{u})'$ is $0$, i.e., $u$ and $v$ are linearly dependent.
\paragraph{Example}
$$Lu = u''$$
with periodic bound conditions on $[-\pi, \pi]$. Solutions are $\sin(\theta)$ and $\cos(\theta)$. 
\paragraph{Projection}
Suppose $u_i$ are normalized eigenfunctions of $L$, and
$$V_n = \text{span}\qty(u_1, \dots, u_n)$$
If $f\in X$, what is closest function to $f$ in $V_n$:
$$\min\limits_{w\in V_N} \norm{w-f}$$

We call $w$ a projection of $f$ on $V_N$.
\paragraph{Claim} 
$$w = \sum_i^N \langle f, u_i\rangle u_i \in V_N$$
\subparagraph{Proof}
$$w\in V_N \Rightarrow w = \sum_1^N \alpha_i u_i$$
$\alpha_1, \dots, \alpha_N$ fulfill
$$\min\limits_{\alpha_1, \dots, \alpha_N} \norm{f-\sum_1^N \alpha_i u_i}^2$$
\begin{align*}
\norm{f-\sum_1^N \alpha_i u_i}^2 = \left\langle  f-\sum_1^N \alpha_i u_i,f-\sum_1^N \alpha_i u_i\right\rangle = \left\langle  f,f\right\rangle - 2 \sum \alpha_i \left\langle  f,u_i\right\rangle + \sum_{i,j} \alpha_i \alpha_j \left\langle  u_i,u_j\right\rangle =\\= \norm{f}^2 -2\alpha_i  \left\langle  f,u_i\right\rangle + \sum \alpha_i^2 = H(\alpha_i, \dots, \alpha_N)
\end{align*}
$$$$
$$\pdv{H}{\alpha_i} = 2[\alpha_i - \left\langle  f,u_i\right\rangle ] = 0$$
Thus
$$\alpha_i = \left\langle  f,u_i\right\rangle$$
\paragraph{Bessel inequality}
Let $\qty{u_i}$ -- infinite sequence of orthonormal functions in $X$, for all $f\in X$
$$\sum_i \abs{\left\langle  f,u_i\right\rangle}^2 \leq \norm{f}^2$$

$$0 \leq \norm{f- \sum_1^N \left\langle  f,u_i\right\rangle u_i }^2 = \norm{f}^2 - \sum_{i=1}^N  \abs{\left\langle  f,u_i\right\rangle}^2$$
If
$$\sum_{i=1}^\infty \abs{\left\langle  f,u_i\right\rangle}^2 = \norm{f}^2$$
Then
$$\lim_{N\to \infty} \norm{f- \sum_1^N \left\langle  f,u_i\right\rangle u_i } = 0$$
and we say that series
$$\sum_i^\infty \langle f, u_i\rangle u_i = f$$
Note this is not pointwise convergence:
$$\sum_i^\infty \langle f, u_i\rangle u_i(x) \neq f(x)$$
but rather norm convergence:
$$\lim_{N \to \infty} \int_a^b \abs{f(x) - \sum_i^\infty \langle f, u_i\rangle u_i(x)}^2 r(x) \dd{x}$$

\paragraph{Parseval's identity}
Orthonormal sequence is called total if for all $f\in X$ 
$$\sum_{i=1}^\infty \abs{\left\langle  f,u_i\right\rangle}^2 = \norm{f}^2$$
\paragraph{Theorem}
\begin{enumerate}
	\item Sequence of eigenfunctions of  Sturm–Liouville is total sequence.
	\item Eigenvalues are $\lambda_1 < \lambda_2 < \dots < \lambda_n < \dots$ and $\lim_{n\to \infty} \lambda_n = \infty$.
\end{enumerate}
\paragraph{Conclusion}
If $f\in X$ then series 
$$\sum_{n=1}^{\infty} \left\langle  f,u_i\right\rangle u_i = \sum_{n=1}^\infty \frac{2}{\pi} \qty[\int_0^\pi f(y) \sin(ny) \dd{y}] \sin(nx)$$ 
$$\left\langle  f,u_i\right\rangle = \sqrt{\frac{2}{\pi}} \int_0^\pi f(x) \sin(nx) \dd{x}$$
$$a_n(f) = \frac{2}{\pi } \int_0^\pi f(x) \sin(nx) \dd{x}$$
$$f(x) = \sum a_n \sin(n x)$$
\paragraph{Theorem} 
$$l = \int_a^b \sqrt{\frac{r(x)}{p(x)}} \dd{x}$$
$$\lambda_n \sim \qty(\frac{n\pi}{l})^2$$
which means
$$\lim_{n\to \infty} \lambda_n \sim \qty(\frac{l}{n\pi})^2 = 1$$
In addition, exist constants
$$\abs{u_n}<c$$
$$\abs{u'_n}<cn$$
$$\abs{u''_n}<cn^2$$
\paragraph{Generalized heat equation}
$$\begin{cases}
r(x)u_t = (pu_x)_x + qu\\B_a(u)= B_b(u) = 0\\u(x,0)=  f
\end{cases}$$
$$(pu_x^{(i)})_x + qu^{(i)} + r\lambda_i ^{(i)} = 0$$
Lets search the solution of form
$$u(x,t) = \sum_{j=1}^\infty a_j(t) u^{(j)}(x)$$
$$r(x)  \sum_{j=1}^\infty \dot{a}_j(t) u^{(i)}(x) = r(x)  \sum_{j=1}^\infty a_j(t) \lambda_j u^{(j)}(x)$$
i.e.,
$$\dot{a}_j(t) = \lambda_j a_j(t)$$
$$a_j(t) = a_j(0) e^{-\lambda_j t}$$
Thus
$$u(x,t) = \sum_{j=1}^\infty a_j(0) e^{-\lambda_j t} u^{(j)}(x)$$
$a_j(0)$ are generalized Fourier coefficients of $f$:
$$a_j(0) = \int_a^b rfu^{(j)} \dd{x}$$
$$u(x,t) = \sum_{j=1}^\infty \qty(\int_a^b r(y)f(y) u^{(j)}(y) \dd{y})e^{-\lambda_j t} u^{(j)}(x) = \int_a^b \qty[\sum_{j=1}^\infty r(y) u^{(j)}(y) e^{-\lambda_j t} u^{(j)}(x) ]f(y)\dd{y} $$
Define kernel of heat
$$K(x,y,t) = r(y)\sum_{j=1}^\infty u^{(j)}(x) u^{(j)}(y) e^{-\lambda_j t}  $$
And then
$$u(x,t) = \int_a^b K(x,y,t) f(y) \dd{y}$$
Note that $K$ is symmetric:
$$K(x,y,t) = K(y,x,t)$$
We can say that
$$K(x,y,t) \approxeq e^{Lt}$$
For operator $L$ with eigenvalues $\lambda_n$ and eigenfunctions $u_n$, define
$$\mathcal{L}(x,y) =  \sum_{n=1}^\infty \lambda_n u_n(x) u_n(y)$$
Then
$$Lu = \int \mathcal{L}(x,y) u(y) \dd{y} = \sum_{n=1}^\infty \lambda_n \langle u, u_n \rangle u_n(x)$$
i.e.,
$$Lu_n = \lambda_n u_n$$
Thus
$$e^{\mathcal{L} t} = \sum e^{-\lambda_n t} u_n(x) u_n(y) = K$$

The last question is whether the series converge:
$$\pdv{u}{t} = -\sum_{j=1}^\infty \lambda_j e^{-\lambda_j t} a_j(f) u_j(x)$$ 
$$\pdv[2]{u}{x} = -\sum_{j=1}^\infty e^{-\lambda_j t} a_j(f) u''_j(x)$$
$$a_j(f) = \int_a^b f(x) u_j(x) r(x) \dd{x}$$
$$\abs{a_j(f)}  < c$$
and
$$\sum_{j=1}^\infty a_j^2(f) - \norm{f}^2 < \infty$$
Thus
$$\lim_{j\to \infty} a_j(f) = 0$$
(this is Riemann–Lebesgue lemma).

Since
$$\abs{ \lambda_j e^{-\lambda_j t} a_j(f)} < j^2 e^{-j^2 t} c $$
If $t>0$ coefficient series absolutely converges and thus the solution is in $\mathcal{C}^\infty$.
\paragraph{Non-homogeneous equations}
$$u_t = (pu_x)_x + qu + f(x,t)$$


First of all, let's solve the following ODE:
$$\dot{x} = ax + f(t)$$
The homogeneous solution is
$$x(t) = e^{at}$$
thus
$$x(t) = a_0 e^{at} +e^{at} \int_0^t e^{-as} f(s) \dd{s}$$


We want to solve
$$\dot{a}_j = -\lambda_j a_j + f_j(t)$$
where
$$f_j(t) = \int_a^b f(t) u_j(x) \dd{x}$$
thus
$$a_j(t) = \underbrace{a_j(0)}_{0} + e^{-\lambda_j t} \int_0^t e^{\lambda_j s} f_j(s) \dd{s}$$
$$\abs{ \dot{a}_j(t)} \abs{f_j(t)} + \abs{ \lambda_j} \abs{a_j}$$
Since $\abs{a_j} \sim e^{-j^2}$,we want only
$$\sum_{j=1}^\infty \abs{f_j(t)} < \infty$$
and then
$$\pdv{u}{t} = \sum \dot{a}_j u_j(x) < c$$

If $f$ is differentiable, $\sum_{j=1}^\infty \abs{f_j(t)} < \infty$:
%TODO
\paragraph{Note}