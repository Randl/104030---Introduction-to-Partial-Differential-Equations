If we substitute derivatives back into general equation
\begin{align*}
au_{xx} + 2bu_{xy} + cu_{yy} = a \qty[w_{\xi\xi} \xi_{x}^2 + w_{\xi\eta}\xi_x\eta_x + w_{\xi}\xi_{xx} + w_{\eta \xi}\eta_x\xi_x + w_{\eta\eta} \eta_{x}^2+ w_{\eta}\eta_{xx} ] +\\+
2b\qty[w_{\xi \xi} \xi_x\xi_y + w_{\xi\eta}\xi_y\eta_x + w_{\xi\eta}\xi_x\eta_y + w_{\eta \xi} \eta_x\eta_y + w_{\xi}\xi_{xy}+w_\eta\eta_{xy}] +\\+ 
c\qty[w_{\xi\xi} \xi_{y}^2 + w_{\xi\eta}\xi_y\eta_y + w_{\xi}\xi_{yy} + w_{\eta \xi}\eta_y\xi_y + w_{\eta\eta} \eta_{y}^2+ w_{\eta}\eta_{yy} ]  =\\=
\qty\bigg(a\xi_x^2+3b\xi_x\xi_y + c\xi_y^2)w_{\xi\xi} + 2\qty\bigg(a\xi_x\eta_x + c\eta_y\xi_y + b(\xi_x\eta_y + \xi_y\eta_x))w_{\xi\eta} + \qty\bigg(a\eta_x^2+3b\eta_x\eta_y + c\eta_y^2)w_{\eta\eta} + \dots
\end{align*}
We can rewrite it in matrix form as
$$\begin{pmatrix}
\xi_x& \xi_y \\ \eta_x & \eta_y
\end{pmatrix} \begin{pmatrix}
a&b\\b&c
\end{pmatrix} \begin{pmatrix}
\xi_x& \eta_x\\ \xi_y  & \eta_y
\end{pmatrix} = \begin{pmatrix}
A&B\\B&C
\end{pmatrix}$$
$$\begin{vmatrix}
A&B\\B&C
\end{vmatrix} = \begin{vmatrix}
a&b\\b&c
\end{vmatrix} \cdot \begin{vmatrix}
\xi_x& \eta_x\\ \xi_y  & \eta_y
\end{vmatrix}^2$$
Since the determinant is exactly $ac-b^2$, under the variable substitution the sign of $b^2-ac$ is conserved.

\paragraph{Canonical form}
The form $w_{\xi \eta} + \ell_1[w] = G(\xi, \eta)$, where $\ell_1$ is first-order differential operator is called canonical form of hyperbolic equation.
\paragraph{Theorem} Each hyperbolic equation can be written in canonical form
\subparagraph{Proof}
We want to show that 
$$\begin{cases}
A = a\xi_x^2 + 2b\xi_x\xi_y + c\xi_y^2 = 0\\
C =a\eta_x^2 + 2b\eta_x\eta_y + c\eta_y^2 = 0
\end{cases}$$
i.e., that equation $a\psi_x^2 + 2b\psi_x\psi_y + c\psi_y^2 = 0$ has two independent solutions.

Dividing by $\psi_y^2$:
$$a\left(\frac{\psi_x}{\psi_y}\right)^2 + 2b\frac{\psi_x}{\psi_y} + c = 0$$
This is algebric equation, with solutions
$$\frac{\psi_x}{\psi_y} = \frac{-b\pm \sqrt{b^2-ac}}{a} = \lambda_\pm$$
We acquired a pair of equations
$$\psi_x - \lambda_\pm \psi_y = 0$$
And those are two independent solutions which result in $A=0$ and $C=0$.

\paragraph{Wave equation canonical form}
$$
u_{tt} - c^2 u_{xx} = 0$$
The canonical change of coordinates is
$$\begin{cases}
\xi = x-ct\\
\eta = x+ct
\end{cases}$$
$$u_{t} = -cw_\xi + cw_\eta$$
$$u_{x} = w_\xi + w_\eta$$
$$u_{tt} = c^2 w_{\xi\xi} - 2c^2w_{\xi\eta} +  c^2w_{\eta\eta}$$
$$u_{xx} = w_{\xi\xi} + 2w_{\xi\eta} + w_{\eta\eta}$$
Then
$$u_{tt} - c^2u_{xx} = -4c^2w_{\xi\eta}$$

The solution of canonical equation $w_{\xi \eta} = 0$ is $w(\xi, \eta) = F(\xi) + G(\eta)$, thus solution of wave equation:
$$u(x,t) = F(x+ct) + G(x-ct)$$

An example for physical object fulfilling wave equation is infinite string. To find a solution we need initial conditions, for example, velocity and location at time $t=0$:
$$\begin{cases}
u(x,0) = f(x)\\u_t(x,0) = g(x)
\end{cases}$$
, where $f\in \mathcal{C}^2$, $g\in \mathcal{C}^1$.
\paragraph{Theorem}
Exists unique solution of wave equation with those initial conditions. 
\subparagraph{Proof}
Substituting initial conditions into general solutions:
$$\begin{cases}
u(x,0) = F(x) + G(x) = f(x)\\
u_t(x,0) = c\qty[F'(x) - G'(x) ] = g(x)
\end{cases}$$
$$\begin{cases}
F'(x) + G'(x) = f'(x)\\
F'(x) - G'(x)  = \frac{g(x)}{c}
\end{cases} \Rightarrow \begin{cases}
F'(x) = \frac{f'(x) }{2} + \frac{g(x)}{2c}\\
G'(x)  = \frac{f'(x) }{2} - \frac{g(x)}{2c}
\end{cases} \Rightarrow \begin{cases}
F(x) = \frac{f(x) }{2} + \frac{1}{2c}\int_0^x g(s) \dd{s} + D_1\\
G(x)  = \frac{f(x) }{2} - \frac{1}{2c}\int_0^x g(s) \dd{s} + D_2
\end{cases}$$
Now, since $F(x)+G(x) = f(x)$, thus $D_1 + D_2 = 0$.

Substituting into solution, we acquire what is called d'Alembert's formula:
$$u(x,t) = \frac{f(x-ct) + f(x+ct) }{2} + \frac{1}{2c}\int_{x-ct}^{x+ct} g(s) \dd{s} $$
From construction, the solution is unique.
\paragraph{Example}
$$\begin{cases}
g(x) = 0 \\
f(x) = e^{-x^2}
\end{cases}$$
$$u(x,t) = \frac{1}{2} e^{-(x+ct)^2} + \frac{1}{2}e^{-(x-ct)^2} $$
\paragraph{Standing wave}
To get standing wave we want $G=0$, i.e., 
$$\begin{cases}
f(x) = F(x)\\
g(x) = cF'(x)
\end{cases} \Rightarrow g(x) = cf'(x)$$
\paragraph{Domain of dependence and region of influence}
Domain of dependence of $u$ in point $(x_0,t_0)$ is a characteristic triangle with vertices $(x_0-ct_0,0)$, $(x_0+ct_0, 0)$, $(x_0,t_0)$. Any point outside of triangle doesn't affect the value of $u$ in point.

Region of influence of point $x_0$ is cone bounded by condition $x_0-ct<x<x_0+ct$.
\begin{center}	
	\includesvg[eps,svgpath = lect3/,width=0.5\linewidth]{pic1}
\end{center}
%TODO
\paragraph{Weak solution}
$$\begin{cases}
g(x) = 0\\f(x) = \begin{cases}
1 & |x| <1 \\
0 & |x| > 1
\end{cases}
\end{cases}$$
The weak solution
$$u(x,t) = \frac{1}{2}f(x-ct)+\frac{1}{2}f(x+ct)$$
is not differentiable, but is solves the equation in some sense.
\subsection{Generalization of  d'Alembert's formula for non-homogeneous equations}
Consider non-homogeneous equation
$$u_{tt} - c^2 u_{xx} = \varphi(x,t)$$

Remember Green's theorem, for differentiable $P$ and $Q$ defined in $\Omega$:
$$\iint\limits_\Omega \pdv{Q}{x} - \pdv{P}{t} \dd{x} \dd{t} = \oint\limits_{\partial \Omega} P(x,t) \dd{x} + Q(x,t) \dd{t}$$

Lets define $Q=c^2u_x$ and $P=u_t$, and choose $\Omega(x_0, t_0)$ to be characteristic triangle.

$$\iint\limits_{\Omega(x_0, t_0)} \varphi(x,t) \dd{x} \dd{t} = \iint\limits_{\Omega(x_0, t_0)} u_{tt} - c^2 u_{xx} \dd{x} \dd{t} = \oint P(x,t) \dd{x} - Q(x,t) \dd{t} = - \qty[\oint\limits_{\partial \Omega(x_0, t_0)} u_t \dd{x} + c^2u_x \dd{t}] $$

Lets divide the curve integral into three integrals along each of lines. For first line $\dd{t} = 0$, for second $\dd{x} + c\dd{t} = 0$ and for third $\dd{x} - c\dd{t} = 0$.
\begin{align*}
\oint\limits_{\partial \Omega(x_0, t_0)} u_t \dd{x} + c^2u_x \dd{t} =\\= \int_{x_0-ct_0}^{x_0+ct_0} \underbrace{u_t}_{g(x) \text{ in } t=0} \dd{x} - \int_{(x_0+ct_0,0)}^{(x_0, t_0)} cu_t \dd{t} + u_x \dd{x} +  \int_{(x_0, t_0)}^{(x_0-ct_0,0)} cu_t \dd{t} + u_x \dd{x} =\\= \int_{x_0-ct_0}^{x_0+ct_0} g(s) \dd{s} - c\int_{(x_0+ct_0,0)}^{(x_0, t_0)} \dd{u} + c \int_{(x_0, t_0)}^{(x_0-ct_0,0)} \dd{u} 
\end{align*}
Since
$$\int_{(x_0+ct_0,0)}^{(x_0, t_0)} \dd{u} = u(x_0, t_0) - f(x_0+ct_0) $$
$$ \int_{(x_0, t_0)}^{(x_0-ct_0,0)} \dd{u} = f(x_0-ct_0) -  u(x_0, t_0) $$
we get
$$\iint\limits_{\Omega(x_0, t_0)} \varphi(x,t) \dd{x} \dd{t} = -\int_{x_0-ct_0}^{x_0+ct_0} g(s) \dd{s} + 2cu(x_0, t_0) - cf(x_0+ct_0) - cf(x_0-ct_0) $$
from which we get the solution

$$u(x,t) = \frac{f(x-ct) + f(x+ct) }{2} + \frac{1}{2c}\int_{x-ct}^{x+ct} g(s) \dd{s} + \frac{1}{2c}\iint\limits_{\Omega(x_0, t_0)} \varphi(\xi, \eta) \dd{\xi} \dd{\eta}$$

We've got a cadidate for the solution. Let's check that $u$ is actually solving PDE. Define $v$, $w$, such that $w$ is a solution of homogeneous PDE and $v=u-w$, i.e.,
$$v(x,t) = \frac{1}{2c}\iint\limits_{\Omega(x_0, t_0)} \varphi(\xi, \eta) \dd{\xi} \dd{\eta}$$

Let's show that $v$ solves PDE. Rewrite $v$ as double integral:
$$v(x,t) = \frac{1}{2c}\int_0^t \int_{x-c(t-\tau)}^{x+c(t-\tau)} \varphi(\xi, \tau) \dd{\xi} \dd{\tau}$$
Define
$$H(x,t,\tau) = \int_{x-c(t-\tau)}^{x+c(t-\tau)} \varphi(\xi, \tau) \dd{\xi}$$
and then
$$v(x,t) = \frac{1}{2c} \int_0^t H(x,t,\tau) \dd{\tau}$$
$$\pdv{v}{t} = \frac{1}{2c} \underbrace{H(x,t,t)}_{0} +\frac{1}{2c} \int_0^t \pdv{H}{t}(x,t,\tau) \dd{\tau}$$
$$\pdv{H}{t} =c\qty\Big[\varphi\qty\big(x+c(t-\tau), \tau) + \varphi\qty\big(x-c(t-\tau), \tau) ]$$
$$\pdv[2]{H}{t} =c^2\qty\Big[\varphi_x\qty\big(x+c(t-\tau), \tau) - \varphi_x\qty\big(x-c(t-\tau), \tau) ]$$
$$\pdv[2]{v}{t} =\frac{1}{2c} \int_0^t \pdv[2]{H}{t}(x,t,\tau) \dd{\tau} =  \varphi(x,t) + \frac{c}{2} \int_0^t \varphi_x\qty\big(x+c(t-\tau), \tau) - \varphi_x\qty\big(x-c(t-\tau), \tau)  \dd{\tau} $$
$$\pdv{v}{x} = \frac{1}{2c} \int_0^t \pdv{H}{x} \dd{\tau}$$
$$\pdv{H}{x} = \varphi\qty\big(x+c(t-\tau),\tau) - \varphi\qty\big(x-c(t-\tau),\tau)$$
$$\pdv[2]{H}{x} = \varphi_x\qty\big(x+c(t-\tau),\tau) - \varphi_x\qty\big(x-c(t-\tau),\tau)$$
$$\pdv[2]{v}{x} = \frac{1}{2c} \int_0^t \varphi_x\qty\big(x+c(t-\tau),\tau) - \varphi_x\qty\big(x-c(t-\tau),\tau) \dd{\tau}$$
Thus we got
$$\pdv[2]{v}{t} - c^2\pdv[2]{v}{x} = \varphi(x,t)$$

Suppose we have two solutions $u_1$ and $u_2$ then $u=u_1-u_2$ is solution of homogeneous equation with $0$ initial conditions, and thus $u=0$. That means the solution is unique.

The presented initial condition problem has 3 properties:
\begin{enumerate}
	\item Solution exist
	\item It's unique
	\item It's stable
\end{enumerate}
\paragraph{Stability of wave equation}
For all $\tau > 0 $, $\epsilon >0$, exists $\delta >0$ such that if 
$$\begin{cases}
|f(x) - \tilde{f}(x) | < \delta\\
|g(x) - \tilde{g}(x) | < \delta\\
|\varphi(x) - \tilde{\varphi}(x) | < \delta\\
\end{cases}$$
For all $-\infty < x < \infty$ and $0 \leq t \leq \tau$ and if $u$, $\tilde{u}$ are solutions of corresponding wave equations, then
$$|u(x,t) - \tilde{u}(x,t)| < \epsilon$$
\subparagraph{Proof}
From the general solution:
\begin{align*}
u(x,t) - \tilde{u}(x,t) =\\= \qty|\frac{f(x+ct) + f(x-ct)}{2} - \frac{\tilde{f}(x+ct) + \tilde{f}(x-ct)}{2} + \frac{1}{2c}\int_{x-ct}^{x+ct} g(s) - \tilde{g}(s)\dd{s}  + \frac{1}{2c}\iint\limits_{\Omega(x_0, t_0)} \varphi(\xi, \eta) - \tilde{\varphi}(\xi, \eta)\dd{\xi} \dd{\eta}| \leq\\\leq \qty|\frac{f(x+ct) - \tilde{f}(x+ct)}{2}| + \qty|\frac{f(x-ct) - \tilde{f}(x-ct)}{2}| + \frac{1}{2c}\int_{x-ct}^{x+ct} \qty|g(s) - \tilde{g}(s)|\dd{s}  + \frac{1}{2c}\iint\limits_{\Omega(x_0, t_0)} \qty|\varphi(\xi, \eta) - \tilde{\varphi}(\xi, \eta)|\dd{\xi} \dd{\eta} \leq\\\leq \frac{\delta}{2}+\frac{\delta}{2} \frac{1}{2c} \cdot 2c \cdot \delta + \frac{1}{2c} \frac{ct}{2} = 2\delta + \frac{\delta t}{4} \leq 2\delta + \frac{\delta \tau}{4} \leq \epsilon
\end{align*}
Thus we choose $\delta < \frac{\epsilon}{2+\frac{\tau}{4}}$.