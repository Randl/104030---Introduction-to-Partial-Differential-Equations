If crossing conditions are not fulfilled we have a couple of options:
\begin{itemize}
	\item If initial curve is characteristic curve, we have infinite number of solutions.
	\item If initial curve is not characteristic curve, but their projection on $xy$-plane is same, we have no solution, since each solution includes characteristic curve.
\end{itemize}

In other cases, if for example initial curve is tangent to characteristic curve and their projection on $xy$-plane are different, there are different possibilities.

\paragraph{Example}
$$yu_x - xu_y = 0$$
with initial curve $(s,0,H(s))$ and $0<\alpha\leq s\leq \beta$
\subparagraph{Solution}
$$\begin{cases}
\dot{x} = y\\
\dot{y} = -x\\
\dot{z} = 0
\end{cases} \Rightarrow \begin{cases}
\ddot{x} = -x\\
\ddot{y} = -y\\
\dot{z} = 0
\end{cases} \Rightarrow \begin{cases}
x = A(s)\sin(t)+B(s)\cos(t)\\
y = A(s)\cos(t)-B(s)\sin(t)\\
z = c
\end{cases}$$
Now from initial curve
$$A(s)= s \quad B(s) = 0$$
$$ \begin{cases}
x = s\cos(t)\\
y = -s\sin(t)\\
z = h(s)
\end{cases}$$
Now we want to find $s$, $t$ as a function of $x$,$y$:
$$x^2+y^2 = s^2 \Rightarrow s = \sqrt{x^2+y^2}$$
$$u(x,y)  = h\left(\sqrt{x^2+y^2}\right)$$
Note that characteristic curves are rings.

\paragraph{Example}
$$yu_x - xu_y = u$$
with initial curve $(s,0,H(s))$ and $0<\alpha\leq s\leq \beta$
\subparagraph{Solution}
$$\begin{cases}
\dot{x} = y\\
\dot{y} = -x\\
\dot{z} = u
\end{cases} \Rightarrow \begin{cases}
\ddot{x} = -x\\
\ddot{y} = -y\\
z = C(s)e^t
\end{cases} \Rightarrow \begin{cases}
x = A(s)\sin(t)+B(s)\cos(t)\\
y = A(s)\cos(t)-B(s)\sin(t)\\
z = C(s)e^t
\end{cases}$$
Now from initial curve
$$A(s)= s \quad B(s) = 0$$
$$ \begin{cases}
x = s\cos(t)\\
y = -s\sin(t)\\
z = h(s)e^t
\end{cases}$$
Now we want to find $s$, $t$ as a function of $x$,$y$:
$$x^2+y^2 = s^2 \Rightarrow s = \sqrt{x^2+y^2}$$
Now
$$\tan t  = -\frac{y}{x} \Rightarrow t = \arctan \left(-\frac{y}{x}\right)$$
$$u(x,y)  = h\left(\sqrt{x^2+y^2}\right)e^{ \arctan \left(-\frac{y}{x}\right)}$$


\subsection{Burgers' equation}
$$u_y + uu_x = 0$$
(which is partial case of equation of form $$\frac{\partial u}{\partial y} + \frac{\partial }{\partial y} F(u) = 0$$ for $F=\frac{1}{2}u^2$)

Note that $$\frac{u_y}{u_x} = -u \Rightarrow \frac{dx}{dy} = - u \Rightarrow u = \frac{dx}{dy}$$
Here $y$ denotes time.

To solve it, we take integral:
$$\int_a^b \left[ \frac{\partial u(x,y)}{\partial y} + \frac{\partial }{\partial x} F(u(x,y))\right] dx = 0$$
$$\frac{\partial }{\partial y} \underbrace{\int_a^b u dx}_{Q(y)}  + F\Big(u(b,y)\Big) - F\Big(u(a,y)\Big)$$
$$\frac{dQ}{dy} = F\Big(u(a,y)\Big) - F\Big(u(b,y)\Big)$$

Now as for any quasilinear PDE:
$$\begin{cases}
\dot{x}= z\\\dot{y} = 1\\\dot{z}= 0
\end{cases} \Rightarrow \begin{cases}
x= c_2 t + c_3\\y = t + c_1\\\dot{z}= c_2
\end{cases} $$
For initial conditions $\Big(s,0,h(s)\Big)$:
$$ \begin{cases}
x= h(s) t + s\\y = t \\\dot{z}= h(s)
\end{cases} $$
Now 
$$s=x-yu \Rightarrow u = h(x-yu)$$
Checking transversality condition:
$$\frac{d\bar{x}}{ds} \cdot 1 - \frac{d\bar{y}}{ds} \cdot h(s) = 1 \neq 0$$  


Since
$$\frac{\partial u}{\partial x} = h'(x-yu) \cdot \left(1 - y\frac{\partial u}{\partial x} \right)$$
$$\frac{\partial u}{\partial x} = \frac{h'(x-yu)}{1+h'(x-yu) \cdot y} $$
even if we start from $\mathcal{C}^\infty$ function we can get $1+h'(x-yu) \cdot y = 0$ and thus undefined derivative.

Geometrically, the slope of projections of characteristic curves is equal to $h(s)$ thus they can cross in some point.

\paragraph{Weak solutions}
We define a weak solution of equation, function $u$  fulfilling the equation:
$$\forall a,b \quad \frac{\partial }{\partial y} u(x,y) dx + F\Big(u(b,y)\Big)-F\Big(u(a,y)\Big) = 0$$

Intuitively, $F$ is flux, and $u$ is density, thus change in number of particles (integral) is difference between particles going in and out.

Suppose for solution $u(x,y)$ exists curve of non-continuousness $\gamma$, i.e, $u$ is not continuous in each point of curve:
$$u(y) = \begin{cases}
u^+(y) & y < \gamma(y)\\
u^-(y) & y > \gamma(y)
\end{cases}$$

$$Q_{a,b}(y) = \int_a^b u(x,y) dx = \int_a^{\gamma(y)} u^+(x,y) dx + \int_{\gamma(y)}^{b} u^-(x,y) dx$$ 
\begin{align*}
\frac{\partial Q}{\partial y} = \int_a^{\gamma(y)} \frac{\partial u^+(x,y)}{\partial y} dx + u^+(x,\gamma(y)) \cdot \gamma'(y) + \int_{\gamma(y)}^{b} \frac{\partial u^-(x,y)}{\partial y} dx -  u^-(x,\gamma(y)) \cdot \gamma'(y) =\\= - \int_{a}^{\gamma(y)}  \frac{d F(u^+)}{dx}dx - \int_{\gamma(y)}^{b} \frac{d F(u^-)}{dx}dx + \gamma'(y) \left[ u^+\Big(x,\gamma(y)\Big) -  u^-\Big(x,\gamma(y)\Big) \right] =\\= -\left[F\Big(u^+\big(\gamma(y),y\big)\Big)-F\Big(u^+(a,y)\Big)\right]-\left[F\Big(u^-(b,y)\Big)-F\Big(u^+\big(\gamma(y),y\big)\Big)\right] + \gamma'(y) \left[ u^+\Big(x,\gamma(y)\Big) -  u^-\Big(x,\gamma(y)\Big)  \right]
\end{align*}
Meaning
$$-\bigg[F\Big(u^+\big(\gamma(y),y\big)\Big)-F(u^+(a,y))\bigg]-\bigg[F\Big(u^-(b,y)\Big)-F\Big(u^+\big(\gamma(y),y\big)\Big)\bigg] + \gamma'(y) \bigg[ u^+(x,\gamma(y)) -  u^-(x,\gamma(y))  \bigg] = F\Big(u^-(a,y)\Big) - F\Big(u^+(b,y)\Big)  $$
$$  \gamma'(y) \bigg[ u^+\Big(x,\gamma(y)\Big) -  u^-\Big(x,\gamma(y)\Big)  \bigg] = F\Big(u^+\big(\gamma(y),y\big)\Big) - F\Big(u^-\big(\gamma(y),y\big)\Big)  $$
$$\gamma' = \frac{ F\Big(u^+(\gamma(y),y)\Big) - F\Big(u^-\big(\gamma(y),y\big)\Big)}{u^+\big(x,\gamma(y)\big) -  u^-\big(x,\gamma(y)\big) }$$
This equation is called Rankine–Hugoniot conditions.
If $F(u) = \frac{1}{2}u^2$, we get $\gamma'(y) = \frac{1}{2}\left(u^+ + u^-\right)$

\paragraph{Example}
Suppose we have initial conditions $u(x,0) = h(x)$ for
$$h(x) = \begin{cases}
1 & x < 0\\
0 & x> \alpha\\
1-\frac{x}{\alpha} & 0\leq x \leq \alpha
\end{cases}$$

For $0<y<1$ we have a triangle $\Delta$ ($0<x<\alpha$ and $y < \frac{x}{\alpha}$) for which there is intersection of two solution:

\begin{center}
	\includegraphics[width=0.5\linewidth]{./lect2/p2.png}
\end{center}

In point $x,y$ we have slope $u(x,y)$ thus the charecteristic curve crosses $x$-axis at $x_0 = x-uy$ and from initial conditions, $u=1-\frac{x_0}{\alpha}$. Thus
$$u=1-\frac{x-uy}{\alpha}$$
$$\alpha u=\alpha-x+uy$$
$$(\alpha-y) u=\alpha-x$$
Acquiring
$$u = \frac{x-\alpha}{y-\alpha}$$
And now for $y>1$ from Rankine–Hugoniot conditions
$$u(x,y) = \begin{cases}
1 & x<\alpha + \frac{1}{2} (y-\alpha)\\
0 & x>\alpha + \frac{1}{2} (y-\alpha)\\
\end{cases}$$
Such a solution is called a shock wave.

\begin{center}
	\includegraphics[width=0.7\linewidth]{./lect2/p1.png}
\end{center}

\paragraph{Example}
For
$$h(x) = \begin{cases}
0 & x> \alpha\\
1 & x < 0\\
\frac{x}{\alpha} & 0\leq x \leq \alpha
\end{cases}$$

Now there is no place where characteristic curves meet

\begin{center}
	\includegraphics[width=0.7\linewidth]{./lect2/p3.png}
\end{center}

In the region without characteristic curves ($0\leq x\leq y$) we get the following: the solution starts from some  point $x_0 = x-uy$, and similarly to the previous case, from initial conditions,
$$u=\frac{x}{\alpha +y}$$


\begin{center}
	\includegraphics[width=0.3\linewidth]{./lect2/p4.png}
\end{center}

\paragraph{What happens if $\alpha \to 0$?}
We get $u=\frac{x}{y}$ for $0\leq x \leq y$. We acquired rarefaction wave - starting from something non-continuous we got continuous solution. This is weak solution.

However, also shock wave along $y=x$ is also solution of initial conditions. This solution is worse, because shock wave loses information, which means we cant reproduce the solution for some $y<y_0$ even if I know the values for $y=y_0$.
\paragraph{Entory principle}
Weak solution is unique if characteristic curves meet shock wave from direction of increasing time. 


\subsection{Fully non-linear equations}
\paragraph{Hamilton-Jacoby equation}
$$u_x^2+u_y^2=1$$
can we generalize the method of solution of quasilinear equations to fully non-linear equations? Yes. 
We have some $$F(x,y,u,u_x,u_y) = 0$$. In our case $$F(x,y,u,p,q) = p^2+q^2-1$$

Characteristic equations:
$$\begin{cases}
\frac{dx}{dt} = \frac{\partial F}{\partial p}\\
\frac{dy}{dt} = \frac{\partial F}{\partial q}\\
\frac{dz}{dt} = p\frac{\partial F}{\partial p}+q\frac{\partial F}{\partial q}\\
\frac{dp}{dt} = -\frac{\partial F}{\partial x}+p\frac{\partial F}{\partial z}\\
\frac{dq}{dt} = -\frac{\partial F}{\partial y}+q\frac{\partial F}{\partial z}\\
\end{cases}$$


Suppose we have initial curve $\Gamma = (\bar{x}(s), \bar{y}(s), \bar{z}(s))$

We need to find $\bar{p}$ and $\bar{q}$. We have two additional conditions:
$$F(x,y,u,u_x,u_y) = 0$$
also
$$u(\bar{x}(s), \bar{y}(s)) = \bar{z}(s)$$
Differentiating by $s$
$$\frac{\partial u}{\partial x}\frac{d\bar{x}}{ds} + \frac{\partial u}{\partial y}\frac{d\bar{y}}{ds} = \frac{d\bar{z}}{ds}$$
$$\bar{p}(s)\frac{d\bar{x}}{ds} + \bar{q}(s)\frac{d\bar{y}}{ds} = \frac{d\bar{z}}{ds}$$
Now we can find $p$ and $q$.

Back to our equation:
$$\begin{cases}
\dot{x} = 2p\\
\dot{y} = 2q\\
\dot{z} = 2(p^2+q^2)\\
\dot{p} = \dot{q} = 0
\end{cases}$$

In case we have initial curve with $u=0$, then characteristic curves are perpendicular to initial curve.  We get $u(x,y)$ equal to distance from initial curve, since absolute value of gradient of $u$ is $1$ due to equation.

If we have $u=\phi(s)$ on initial curve, we acquire
$$u(x,y) = \min (x-\bar{x}(s))^2 + (y-\bar{y}(s))^2 + \phi(s)$$

\paragraph{Higher dimension}
We can trivially extend quasilinear equations to more dimensions. In this case we have initial surface instead of curve.
\section{Wave equation}
$$u_{yy} -c^2u_{xx} = 0$$
More generally the equation is
$$a(x,y) u_{xx} + 2b(x,y) u_{xy} + c(x,y) u_{yy} + du + f$$
We want to simplify the equation: we are searching for $s(x,y)$ and $t(x,y)$ and solution $u(x,y) = w(s(x,y), t(x,y)$.

Derivatives of $u$ are
$$u_y = w_s s_y + w_tt_y$$
$$u_{yy} = w_{ss} s_{y}^2 + w_{st}s_yt_y + w_{s}s_{yy} + w_{ts}t_ys_y + w_{tt} t_{y}^2+ w_{t}t_{yy}  $$

Now we can get equation of form
$$A(s,t) w_{ss} + 2B(s,t) w_{st} + C(s,t) w_{tt} + D(s,t) w_{tt}+ F(s,t)$$
If we can find variable substitution such that
$$A=C=D=F=0$$
Then
$$Bw_{st} = 0$$
i.e.,
$$w(s,t) = f(s)+g(t)$$